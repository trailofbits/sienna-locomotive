Sienna Locomotive is a fuzzing and crash triage system with usability features that are intended to attract a wider user base than conventional fuzzing tools. The goal is to bring all the power that software fuzzing has to offer into the hands of less-\/experienced users. Focusing on ease of use won\textquotesingle{}t stop us from trying to make it smarter and faster than the competition.

\subsection*{Getting Started}

\subsubsection*{The project}

Skim the following documents to familiarize yourself with the project\+:
\begin{DoxyItemize}
\item Watch the \href{https://drive.google.com/open?id=1njGgRrrfNanYSuaMy5nwLi1rw2bS1rMt}{\tt Demo video}
\item \href{https://docs.google.com/document/d/1zTUHlu-y_ZLT08saJp0qguYXC69F6CskMPZVLs48IVc/edit}{\tt Interim Technical Report}
\item \href{https://docs.google.com/document/d/1RwvknJk9PPgecLcsQI1SiXje9SdKB3OuOoSniIDvy68/edit}{\tt S\+L2 Reinception Doc}
\item harness.\+py (if you\textquotesingle{}re familiar with Python)
\item Read through the \href{https://github.com/trailofbits/sienna-locomotive/projects/6}{\tt Projects} and \href{https://github.com/trailofbits/sienna-locomotive/issues}{\tt Issues} pages on Git\+Hub
\end{DoxyItemize}

\subsection*{High level architecture}


\begin{DoxyCode}
Wizard
  Run the target program once
  Find all the functions the fuzzer can target
  Return a list the user can select from

Fuzzing harness
  Run the target program
  Hook attack surface functions
  Execute program with mutated inputs
  Catch crashes for tracer

Triage
  Traces and performs taint tracking on a crashing execution
  Combines data about crash, inputs, and taint
  Output score and JSON info about crash

Mutation Server
  Sent bytes by fuzzing harness
  Figures out mutations for bytes
  Sends back mutations
  Responsible for logging information about corrupted inputs

Python harness
  Handles user input and configuration
  Runs all the other components
\end{DoxyCode}


\subsubsection*{Building}

First, clone the repository\+: {\ttfamily git clone \href{https://github.com/trailofbits/sienna-locomotive.git}{\tt https\+://github.\+com/trailofbits/sienna-\/locomotive.\+git}} (or download a zip)

\paragraph*{Dependencies}

Next, install the following dependencies
\begin{DoxyItemize}
\item \href{https://cmake.org/download/}{\tt C\+Make}
\item Visual Studio 2017 (Install components for Windows Console dev)
\item Dynamo\+R\+IO (Automatically installed with make.\+ps1)
\item Python (3.\+6+)
\end{DoxyItemize}

\paragraph*{Build Commands}

Then, from the root of the Sienna Locomotive repository\+: In powershell, run make.\+ps1\+:

{\ttfamily PS C\+:\textbackslash{}proj\textbackslash{}sl2\textbackslash{}sienna-\/locomotive$>$ .\textbackslash{}make.\+ps1}

This should download and install Dynamo\+R\+IO in the sienna-\/locomotive directive if it does not already exist. It will then compile the project.

If you\textquotesingle{}re not familiar with cmake -\/ the first invocation configures it, the second compiles the project. To recompile, just run the final command again.

You might need to use the Visual Studio Developer Command Prompt in order for cmake to be able to see the VS compiler.

Use {\ttfamily .\textbackslash{}make.\+ps1 clean} for a clean build .

Usage {\ttfamily make.\+ps1 help} for more info.


\begin{DoxyCode}
PS C:\(\backslash\)proj\(\backslash\)sl2\(\backslash\)sienna-locomotive> .\(\backslash\)make.ps1 help
Usage: make1.ps [clean|dep|reconfig|help]

make1.ps without any options will build

clean
    Cleans build directory and configuration (reconfigs)

dep
    Rebuild dependencies

reconfig
    Deletes fuzzkit directory with run configuration

help
    This info
\end{DoxyCode}


\subsubsection*{Configuring}

Open up powershell in the project root and run harness.\+py for the first time.

{\ttfamily PS C\+:\textbackslash{}proj\textbackslash{}sl2\textbackslash{}sienna-\/locomotive$>$ python .\textbackslash{}harness.\+py}

\+:warning\+: Python version 3 is required, although on some systems this could be {\ttfamily python.\+exe} or {\ttfamily python3.\+exe} .

It\textquotesingle{}ll create a default configuration file in {\ttfamily A\+P\+P\+D\+A\+TA\textbackslash{}Trail of Bits\textbackslash{}fuzzkit\textbackslash{}}. You can leave everything in there and it will work, but you might want to update the file paths to be relative to C\+: so that you can invoke it from anywhere. If you want to create a new profile, just copy the default one and change the name. Then you can use the {\ttfamily -\/p} flag to harness.\+py to change which profile it pulls settings from.

Using -\/h on the harness will print out the list of command line options it supports. You can set a number of things permanently by adding lines to the configuration file. As a general rule though, the command line parameters will overwrite what\textquotesingle{}s in the config file if you explicitly pass them in. This isn\textquotesingle{}t the case for everything, so if a command isn\textquotesingle{}t working the way you expect, run fuzzer\+\_\+config.\+py with the same arguments to see exactly what settings are getting passed to the harness.

\subsubsection*{Running}

\paragraph*{Via the G\+UI}

{\ttfamily python3 gui.\+py} will run the Qt frontend for the fuzzer. While it provides a convenient way of invoking the components, it doesn\textquotesingle{}t provide as many configuration options. Fortunately, it accepts most of the same command line arguments as the harness, so you can simply pass these in when you invoke the G\+UI. For example\+: {\ttfamily python3 gui.\+py -\/f 15 -\/i 360} will run the gui such that it invokes the fuzzer with a timeout of 15 seconds for each fuzzing run and a timeout of 360 seconds for each triage run. However, it does N\+OT respect the -\/e flag, nor any of the flags that would be overwritten from the config file if invoked on the harness.

\paragraph*{Via the harness}

{\ttfamily python3 harness.\+py} will run the test application in fuzzing mode. By default, the test application will crash after a few fuzzing attempts, so if it doesn\textquotesingle{}t do so when you need it to, you can pass {\ttfamily -\/a 0} to the harness (as the last argument) and it will crash every time. Play around with the command flags to see what else you can do.

\paragraph*{To run individual components manually}

From the root of the project -\/ 
\begin{DoxyCode}
# General Pattern:
.\(\backslash\)dynamorio\(\backslash\)bin64\(\backslash\)drrun.exe -c build\(\backslash\)client\_name\(\backslash\)Debug\(\backslash\)client.dll [client\_args] --
       C:\(\backslash\)path\(\backslash\)to\(\backslash\)target\_application [target\_args]

# triage
.\(\backslash\)dynamorio\(\backslash\)bin64\(\backslash\)drrun.exe -c build\(\backslash\)tracer\_dynamorio\(\backslash\)Debug\(\backslash\)tracer.dll -- corpus\(\backslash\)win\_asm\(\backslash\)crashes.exe 7

# wizard
.\(\backslash\)dynamorio\(\backslash\)bin64\(\backslash\)drrun.exe -c build\(\backslash\)wizard\(\backslash\)Debug\(\backslash\)wizard.dll --
       build\(\backslash\)corpus\(\backslash\)test\_application\(\backslash\)Debug\(\backslash\)test\_application.exe 0

# server
build\(\backslash\)server\(\backslash\)Debug\(\backslash\)server.exe

# fuzzer
.\(\backslash\)dynamorio\(\backslash\)bin64\(\backslash\)drrun.exe -c build\(\backslash\)fuzz\_dynamorio\(\backslash\)Debug\(\backslash\)fuzzer.dll --
       build\(\backslash\)corpus\(\backslash\)test\_application\(\backslash\)Debug\(\backslash\)test\_application.exe 0 -f

# triage crash
.\(\backslash\)dynamorio\(\backslash\)bin64\(\backslash\)drrun.exe -c sienna-locomotive\(\backslash\)build\(\backslash\)tracer\_dynamorio\(\backslash\)Debug\(\backslash\)tracer.dll -r [RUN\_ID] --
       build\(\backslash\)corpus\(\backslash\)test\_application\(\backslash\)Debug\(\backslash\)test\_application.exe 0 -f

# targeting
.\(\backslash\)dynamorio\(\backslash\)bin64\(\backslash\)drrun.exe -c build\(\backslash\)fuzz\_dynamorio\(\backslash\)Debug\(\backslash\)fuzzer.dll -t 0,ReadFile --
       build\(\backslash\)corpus\(\backslash\)test\_application\(\backslash\)Debug\(\backslash\)test\_application.exe 0 -f

.\(\backslash\)dynamorio\(\backslash\)bin64\(\backslash\)drrun.exe -c build\(\backslash\)tracer\_dynamorio\(\backslash\)Debug\(\backslash\)tracer.dll -r [RUN\_ID] -t 0,ReadFile --
       build\(\backslash\)corpus\(\backslash\)test\_application\(\backslash\)Debug\(\backslash\)test\_application.exe 0 -f
\end{DoxyCode}


\#\#\#\# Regression Test 
\begin{DoxyCode}
python.exe .\(\backslash\)regress.py
test\_main (\_\_main\_\_.Test1) ... ok

----------------------------------------------------------------------
Ran 1 test in 0.906s

OK
\end{DoxyCode}


\subsection*{Triage}

The triage system is a separate executable, {\ttfamily triager.\+exe} that is run by the harness. It takes care of ranking exploitability, uniqueness, and binning of crashes.

\subsubsection*{Exploitability}

The Exploitability ranking is a score for the potential ability to exploit a crash based on 3 engines. The ranks, ranging from High (4) to None (0), in order of likelyhood are\+:


\begin{DoxyItemize}
\item {\bfseries High} (4)\+: The mostly likely case of a crash being exploitable.
\item {\bfseries Medium} (3)\+: Between High and Low.
\item {\bfseries Low} (2)\+: At or above the cutoff for low exploitability.
\item {\bfseries Unknown} (1)\+: Unknown cases are below the cutoff for low, but still have the potential to be of interest.
\item {\bfseries None} (0)\+: Very unlikely the crash is exploitable.
\end{DoxyItemize}

\paragraph*{Engines}


\begin{DoxyItemize}
\item {\bfseries Google\textquotesingle{}s Breakpad}\+: This engine uses Google\textquotesingle{}s Breakpad library, which parses minidump files and return an exploitability between High and None as well.
\item {\bfseries Microsoft\textquotesingle{}s {\ttfamily !exploitable}}\+: A reimplementation and approxmiation of the {\ttfamily !exploitable} command for {\ttfamily windbg}, built on top of breakpad.
\item {\bfseries S\+L2 Tracer}\+: Uses the score from our own S\+L2 tracer, which takes taint information into consideration.
\end{DoxyItemize}

\subsubsection*{triage.\+json}

After the tracer has been run, {\ttfamily triager.\+exe} is run on the minidump file. It also loads any information generated by the tracer, and outputs the following json\+:


\begin{DoxyCode}
\{
    \textcolor{comment}{// This is the called functions before the crash}
    \textcolor{stringliteral}{"callStack"}: [
        140699242310037,
        140718144357416,
        140718144581792,
        140718144447545
    ],

    \textcolor{comment}{// The offending memory address}
    \textcolor{stringliteral}{"crashAddress"}: 140699242310037,

    \textcolor{comment}{// The reason of exception type}
    \textcolor{stringliteral}{"crashReason"}: \textcolor{stringliteral}{"EXCEPTION\_BREAKPOINT"},

    \textcolor{comment}{// Exploitability from High to None}
    \textcolor{stringliteral}{"exploitability"}: \textcolor{stringliteral}{"Unknown"},

    \textcolor{comment}{// The instruction pointer at the time of the crash}
    \textcolor{stringliteral}{"instructionPointer"}: 14757395258967641292,

    \textcolor{comment}{// Path to the minidump analyzed}
    \textcolor{stringliteral}{"minidumpPath"}: \textcolor{stringliteral}{"C:\(\backslash\)\(\backslash\)Users\(\backslash\)\(\backslash\)IEUser\(\backslash\)\(\backslash\)AppData\(\backslash\)\(\backslash\)Roaming\(\backslash\)\(\backslash\)Trail of Bits\(\backslash\)\(\backslash\)fuzzkit\(\backslash\)\(\backslash\)runs\(\backslash\)\(\backslash\)
      78f20c60-eb12-410a-8378-342c3afec986\(\backslash\)\(\backslash\)initial.dmp"},

    \textcolor{comment}{// Rank, or numeric version of exploitability from 0-4}
    \textcolor{stringliteral}{"rank"}: 1,

    \textcolor{comment}{// The ranks generated by each of the 3 engines}
    \textcolor{stringliteral}{"ranks"}: [
        0,
        0,
        1
    ],

    \textcolor{comment}{// A unique identifier for the crash. The algorithm uses 12 bits from the called functions,}
    \textcolor{comment}{// and is unaffected by ASLR, function call order, or function call count}
    \textcolor{stringliteral}{"crashash"}: \textcolor{stringliteral}{"f96808cfc4798256"},

    \textcolor{comment}{// Stack pointer at time of crash}
    \textcolor{stringliteral}{"stackPointer"}: 14757395258967641292,

    \textcolor{comment}{// Unique tag for the crash for binning purposes}
    \textcolor{stringliteral}{"tag"}: \textcolor{stringliteral}{"Unknown/EXCEPTION\_BREAKPOINT/f96808cfc4798256"},

    \textcolor{comment}{// Complete output from the tracer run}
    \textcolor{stringliteral}{"tracer"}: \{
        \textcolor{stringliteral}{"exception"}: \textcolor{stringliteral}{"EXCEPTION\_BREAKPOINT"},
        \textcolor{stringliteral}{"instruction"}: \textcolor{stringliteral}{"int3"},
        \textcolor{stringliteral}{"last\_calls"}: [
            140699242861232,
            140699242861064,
            140699242861064,
            140699242861056,
            140699242861184
        ],
        \textcolor{stringliteral}{"last\_insns"}: [
            140699242309722,
            140699242309725,
            140699242309727,
            140699242309730,
            140699242310037
        ],
        \textcolor{stringliteral}{"location"}: 140699242310037,
        \textcolor{stringliteral}{"reason"}: \textcolor{stringliteral}{"breakpoint"},
        \textcolor{stringliteral}{"regs"}: [
            \{
                \textcolor{stringliteral}{"reg"}: \textcolor{stringliteral}{"rax"},
                \textcolor{stringliteral}{"tainted"}: \textcolor{keyword}{false},
                \textcolor{stringliteral}{"value"}: 1080890113
            \},
            \textcolor{comment}{//...............................................}
        ],
        \textcolor{stringliteral}{"score"}: 25,
        \textcolor{stringliteral}{"tainted\_addrs"}: [
            \{
                \textcolor{stringliteral}{"size"}: 8,
                \textcolor{stringliteral}{"start"}: 2645403054665
            \}
        ]
    \}
\}
\end{DoxyCode}


\subsection*{File Formats}

\subsubsection*{F\+KT Format}

This format is for recording mutation events for later replay.

`char magic\mbox{[}4\mbox{]} == \textquotesingle{}F\+KT'\`{}

{\ttfamily uint type}, {\ttfamily 1 == file}

Variable based on type.

{\ttfamily uint file\+\_\+size}

{\ttfamily wchar\+\_\+t path\mbox{[}file\+\_\+size\mbox{]}}

{\ttfamily size\+\_\+t position}

{\ttfamily size\+\_\+t size}

{\ttfamily uchar data\mbox{[}size\mbox{]}}

There\textquotesingle{}s a 010 template for this in {\ttfamily misc}

\subsubsection*{Outdated components}

You can safely ignore most of the stuff in {\ttfamily corpus/asm} and {\ttfamily electriage}

\section*{Changes}

\subsection*{20180808}

Changed passing of arguments to clients and target applications from using comma separated to just normally how it would appear on the command line. The {\ttfamily shlex.\+split()} function will split them up appropriately

\section*{Developer Information}

If you change anything that would break backwards compatibility, increment {\ttfamily harness.\+config.\+V\+E\+R\+S\+I\+ON}. This includes any database changes, formats, directory structures, filenames etc.. 