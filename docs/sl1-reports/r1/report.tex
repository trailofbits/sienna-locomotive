\documentclass{article}

\usepackage{graphicx} 
\usepackage{amsmath} 
\usepackage{url}

\setlength\parindent{0pt}

\renewcommand{\labelenumi}{\alph{enumi}.}
\title{SIENNA LOCOMOTIVE Progress Report} 

\author{Andrew Ruef \\ Trail of Bits} 

\date{\today} 

\begin{document}

\maketitle 

\section{Overview}

Our goal is to create a system that, given a crashing input for a program, 
provides a comprehensible metric for the exploitability of the crash. 
We have reviewed existing research, met with the Army to kick the program 
off, and identified staff within Trail of Bits who will carry out the work.

\section{Research}

Our initial research has focused along two paths: what existing work 
has been done in an industrial setting, and what can a 
notion of \textbf{influence} do when judging the exploitability of crashes?

\subsection{Industrial Research}
We have identified existing literature from the field of crash analysis
and severity ranking \cite{skape} \cite{kratkiewicz2006taxonomy}. 
This work is useful because 
it tells us where things currently are. Work from Matt Miller at Microsoft
provides a useful road map for classifying different types of program faults
when crashes have been found. Microsoft has significant investment into 
fuzzing and fuzzing architectures and they review crashes from multiple
sources to assess severity when deciding how to allocate development 
resources. Understanding their process will take the system closer to 
best-in-class.

\subsection{Influence}
Information flow theory, an extension of taint tracking, can be used for 
fuzzing and crash triage. In information flow theory, data in the system 
can be given an integrity label based on data sensitivity. High 
integrity data is sensitive data in the system, while low integrity data 
is data introduced from outside the system. An analysis on traces can 
then search for violations of noninterference - when low integrity 
data interferes with high integrity data in a manner that can be 
observed by an attacker. This interference and subsequent observation 
could result in the compromise of the data by the attacker. Information 
flow theory is directly applicable to finding and mitigating side-channel 
attacks in cryptosystems\cite{molnar}.

We are hopeful that this notion of influence can produce more actionable
results from a fuzzing campaign. In an information assurance setting it is 
important to provide a notion of scale and severity for a given list of 
crashes and having a precise metric for the number of bits in a crashing state
are under an attackers control could communicate urgency to development 
teams in a way that crashing inputs alone could not.

\section{Execution}

\subsection{Existing tools}
For an unrelated project we have adopted the open source symbolic execution 
system PySymEmu\cite{pysymemu} to operate on mini-dump files produced by 
Windows debugging tools. We're hopeful that this infrastructure is useful when 
reasoning about the exploitability of a given crash. 

\subsection{Goals}
Our goal by the end of the period of performance is to have a prototype that
can take a crashing input or interaction sequence/script, and a dump of the
crashing state as identified by the fuzzer, and produce a fine grained 
severity ranking for that crashing input/interaction. This ranking will 
hopefully incorporate the notion of influence discussed above but perhaps
our research will discover another metric that is better.

\subsection{Meetings}
Andrew Ruef, Yan Ivnitskiy and Nicholas DePetrillo met with Army 
sponsors on date to discuss the program. From the Army side the meeting was 
joined by Joe Law, Roy Radhika, William Daddario, and Rex Johnson. After this meeting Andrew Ruef met with 
William Daddario and Rex Johnson to explore the existing system. This conversation was very 
useful in giving Trail of Bits a view of the needs of the Information 
Assurance group and their existing capabilities. 

\section{Programatics}

We have identified the Trail of Bits staff that will perform the work on 
this project and introduced them to staff at the Army sponsor organization. 
Yan Ivnitskiy will begin work on the program within either the 
next reporting period or the following reporting period. 

\section{Future}

Soon we hope to show a proof of concept from our PySymEmu tool where the 
tool can take a concrete input and, at a crashing condition, demonstrate the
degree of freedom, represented as a constraint, in the pointer dereferenced 
at the crashing location. This work can begin when Yan Ivnitskiy becomes
available. 

\bibliographystyle{acm}
\bibliography{report}

\end{document}
